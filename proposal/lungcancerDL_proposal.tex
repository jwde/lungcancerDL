\documentclass[twocolumn,10pt]{article}
\usepackage[T1]{fontenc}
\usepackage[utf8]{inputenc}
\usepackage[english]{babel}
\usepackage{biblatex}
\usepackage[margin=1in]{geometry}
\setlength{\columnsep}{33pt}
\setcounter{secnumdepth}{0}
\usepackage{minipage-marginpar}
\newcommand\vfilbreak[1]{\vskip 0pt plus #1 \penalty-200 \vskip 0pt plus -#1}
\newenvironment{mpmp}[1]
               {\begin{minipagewithmarginpars}{#1}}
               {\end{minipagewithmarginpars}}
\usepackage{morefloats}
\usepackage{booktabs}

\title{Lung Cancer detection with U-Net/Faster R-CNN \\
nodule/region of interest proposal}
\author{Jay Destories, Jason Fan, Alex Tong}
\date{March 2017}
\renewcommand{\floatpagefraction}{0.85}
\renewcommand{\textfraction}{0.2}

\begin{document}

\maketitle
\section{Introduction and \\Problem~Statement}
The problem with segmentation and structure identification within the field of 
biomedical imaging has become a well developed and very active field in the past
years. In 2016 and 2015, the LUNA challenge and SPIE Lungx challege
\section{Datasets}
\section{Method and Pipeline}
\section{About us}
\end{document}