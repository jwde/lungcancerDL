\documentclass[twocolumn,10pt]{article}
\usepackage[T1]{fontenc}
\usepackage[utf8]{inputenc}
%\usepackage[english]{babel}
\usepackage{biblatex}
\usepackage[margin=1in]{geometry}
\setlength{\columnsep}{33pt}
\setcounter{secnumdepth}{1}
\usepackage{minipage-marginpar}
\newcommand\vfilbreak[1]{\vskip 0pt plus #1 \penalty-200 \vskip 0pt plus -#1}
\newenvironment{mpmp}[1]
               {\begin{minipagewithmarginpars}{#1}}
               {\end{minipagewithmarginpars}}
\usepackage{morefloats}
\usepackage{booktabs}
\usepackage{color}

\title{Lung Cancer detection with U-Net/Faster R-CNN \\
nodule/region of interest proposal}
\author{Jay Destories, Jason Fan, Alex Tong}
\date{March 2017}

\newcommand{\temp}[1]{{\color{red}#1\\}}

\begin{document}

\maketitle
\section{Introduction and \\Problem~Statement}
The problem with segmentation and structure identification within the field of 
biomedical imaging has become a well developed and very active field in the past
years. In 2016 and 2015, the LUng Nodule Annotation (LUNA) challenge and 
SPIE Lungx challege, asked researchers to develop models to identify pulmonary 
nodules in lung CT slices. In the 2016 LUNA Challange, researchers gained access
to annotated CT slices that identified abnormal nodules but did not release data
about the malignancy of the nodles. In the 2015 SPIE less than 80 CT annotated 
images of malignant nodules were released to the public.

However, with the 2017 Kaggle Data Science Bowl, a large\temp{ish} dataset of
1000+ lung CT images in DICOM format was finally released with cancer/no cancer 
labels. Preliminary investigations by Kaggle members
using 3D-convolutional neural nets have already begun. There is one caveat to this 
dataset; location of malignant nodules are not labeled. We seek to present a 
novel pipeline for cancer detection by using techniques employed by region 
proposal networks (RNN) and 
biomedical image segmentation networks to extract high-risk regions to use for
classification.

\temp{Goals:?}
\temp{Tuning Faster/fast RNN or YOLO to detect nodules}
\temp{Using Alex-net or other 2GPU methods to train cancer classifier}
\temp{Find ways to reduce false positive/false negatives}

\section{Datasets}
The number of lung CT scans available to us is very low. The total number of examples
available are many orders of magnitudes smaller than the size of datasets for modern,
state of the art classficiation challenges such as ImageNet or MSCOCO. 

Listed below are the datsets we will leverage to train our classification/nodule
extraction model.

\begin{center}
\begin{tabular}{lll}
  \toprule
  Dataset & \# CT scans & Label Type\\
  \midrule
  Kaggle&1000+&Cancer/No-cancer binary\\
  LIDC-IDRI&888?&Nodule annotation\\
  NLST&?&?\\
  SPIE&80&Nodule annotation\\
\end{tabular}
\end{center}

\subsection{Kaggle Data Science Bowl - (Kaggle)}
\subsection{Lung Image Database Consortium image collection - (LIDC-IDRI)}
\subsection{National Lung Screening Trial - (NLST)}
\subsection{SPIE Lungx Challenge - (SPIE)}


\section{Proposed Method/Algorithm/Pipeline}

\temp{PIPELINE: NODULE EXTRACTION -> CNN -> CANCER CLASSIFICATION}
\temp{Will we even use RNN? If so is UNET an RNN?}
\temp{What kind of augmentations can we do in 3d? in the z axis?}
\section{How we will evaluate our results}
\temp{Visualizations of 3D convolutions?}
\temp{Comparison between plain CNN vs. nodule extraction -> CNN? }
\temp{Comparison between U-NET vs tuned FASTER R-CNN?}
\temp{}
\temp{
	1) Nodule extraction with u-net, r-cnn roi proposal, or another pre-trained
	model. TODO: what will be the extracted features be?! \\
	2) leverage 2 GPU with alex-net like architechture to classify cancer
	instances, we really dont have more data so we might have to train on the
	kaggle dataset.
}

\section{About us}
\temp{buncha kiddos who have no idea about anything}

\section{Related Work}

\section{Questions and challenges}
\begin{enumerate}
	\item What will the output of node-extraction be like?
	\item What kind of novelty will we bring to the table?
	\item How will we deal with the small dataset(s)?
\end{enumerate}
\end{document}