\documentclass[twocolumn,10pt]{article}
\usepackage[T1]{fontenc}
\usepackage[utf8]{inputenc}
%\usepackage[english]{babel}
%\usepackage{biblatex}
\usepackage[margin=1in]{geometry}
\setlength{\columnsep}{33pt}
\setcounter{secnumdepth}{2}
\usepackage{minipage-marginpar}
\newcommand\vfilbreak[1]{\vskip 0pt plus #1 \penalty-200 \vskip 0pt plus -#1}
\newenvironment{mpmp}[1]
               {\begin{minipagewithmarginpars}{#1}}
               {\end{minipagewithmarginpars}}
\usepackage{morefloats}
\usepackage{booktabs}
\usepackage{color}
\usepackage{graphicx}

\title{Lung Cancer Detection with \\Region Enhanced 
Multi-Instance Networks}
\author{Jay Destories, Jason Fan, Alex Tong}
\date{March 2017}

\newcommand{\red}[1]{{\color{red}#1}}
\newcommand{\temp}[1]{{\red{#1}\\}}

\begin{document}

\maketitle
\section{Introduction and \\Problem~Statement}
The problem of image segmentation and structure annotation within the 
field of biomedical imaging has become a well developed and very active field in
the past years. In 2016 and 2015, the LUng Nodule Annotation (LUNA) challenge and 
SPIE Lungx challenge, asked researchers to develop models to identify pulmonary 
nodules in lung CT slices. With the 2016 LUNA Challenge, researchers gained access
to annotated CT slices that contained segmentation ground truths for abnormal 
nodules but did not release data about the malignancy of the nodules. 
In the 2015 SPIE less than 80 CT annotated images of malignant nodules were 
released to the public.

Finally, with the 2017 Kaggle Data Science Bowl, a larger dataset of
1000+ lung CT images in DICOM format was finally released with cancer/no cancer 
labels. This has allowed researchers to answer a deeper question about lung CT
scans; whether or not there are indicators of malignancy and cancer in a patient's
CT scan.

There is, however, one caveat to the Kaggle dataset. 
Although the presence of malignancy is indicated by global, binary 
cancer/no-cancer label, the location of malignant nodules and structures are
\textit{not} annotated in the training data.

Inspired by recent work in biomedical image segmentation
\cite{DBLP:journals/corr/ChristEETBBRAHD16}, region proposal 
networks and multiple instance learning for whole mammogram classification
\cite{Maron:1998:FML:302528.302753},
We seek to present a novel pipeline for lung cancer detection that enhances
multiple instance learning with region proposal.

\section{Related Work}

Over the past decade there has been a significant amount of work towards 
computer aided diagnosis of lung cancer \cite{cad_1998}. Depending on what kind
of data researchers have had access to, previous efforts to identify lung cancer
can be categorized by two main approaches. 

(1) Pulmonary Nodule Detection methods use 
image processing techniques to segment and annotate nodules
\cite{FeatureBasedLungNoduleDetection_2017, 
     LungNoduleDetectionWeaklyLabeled_2016, U-net_2015}. These methods
mimic radiologists by looking for abnormalities in the form of
``solitary white nodule-like blob[s]" in a chest x-rays and CT scans.
Lung nodules are potential cancer indicators, and as such are an important part 
in early lung cancer diagnosis. However, not all pulmonary nodules are malignant
and many are benign, there will be false positives in diagnosis if we naively
associate nodule presence with cancer. 

%\red{HELP I MADE THIS TERM UP}
(2) Direct inference and classification methods
instead attempts to directly predict the probability of cancer using x-ray and
CT images without nodule detection
\cite{Kuruvilla_2013, classificationOfNodules_2016}. \\

One challenge that almost all researchers in biomedical image inference face is 
the problem with datasets being small and weakly labeled. Images labeled with
cancer/no-cancer binaries are weakly labeled because imaged tissues only display
malignancy locally; not \textit{all} of the tissue an image will have cancer. As
Multi-Instance Networks have been used to classify whole mammogram images
\cite{Maron:1998:FML:302528.302753}.

\section{Method/Algorithm/Pipeline}
% \temp{PIPELINE: NODULE EXTRACTION -> CNN -> CANCER CLASSIFICATION}
% \temp{Will we even use RNN? If so is UNET an RNN?}
% \temp{What kind of augmentations can we do in 3d? in the z axis?}


\subsection{Outline of proposed pipeline}
Our Lung Cancer detection method uses nodule annotation to enhance direct
classification. 
\includegraphics[width=\columnwidth]{img/architecture.png}


We plan to leverage state of the art region proposal and biomedical segmentation
to indicate nodule presence and implicitly weight instances consumed by a 
downstream MIL classifier.

In our pipeline, a region proposal network trained on the LUNA dataset to propose
nodule annotations and abnormal regions in CT slices. This region proposal 
is then used to annotate nodules in Kaggle dataset. Then a MIL classifier based 
on the work on mammogram  classification by Lou et. al \cite{DBLP:journals/corr/ZhuLVX16},
will be trained on the annotated Kaggle dataset.

\subsection{Choosing the Region Proposal Network}
We have three region proposal/image segmentation techniques we want to investigate.
We will tune one segmentation/region proposal network for our final pipeline. 

The first segmentation method we will test is U-Net, a CNN for biomedical
segmentation developed by Brox et. al \cite{U-net_2015}. The second segmentation we 
will test is the fully convolutional neural network developed by Darrell et al 
\red{add ref} (Jay and Jason will be investigating this network as a paper 
presentation). And the third region proprosal method we will test is 
Faster R-CNN developed by He et. al \cite{fast_rcnn_2015, faster_rcnn_2015}.

%%%% This paragraph is better in this section, but (jason) can't quite make it flow/work

% In our pipeline we first deal with module detection. For this step there are a 
% number of related papers in particular we will look at U-net for a biologically 
% adapted segmentation algorithm \cite{U-net_2015}. U-net seems to be a fairly 
% simple 2d CNN model that makes most of its gains in data augmentation it is 
% likely that we will be able to use these biologically sound data augmentation 
% techniques with a slightly more sophisticated model that is based on 3d 
% assumptions, something like the 3d CNN trained on weakly labeled data last 
% year at Arizona State University \cite{LungNoduleDetectionWeaklyLabeled_2016}. 
% Region proposal networks such as fast and faster r-cnn may also apply here 
% \cite{fast_rcnn_2015, faster_rcnn_2015}. We will also investigate other 3d 
% segmentation techniques such as the combination of fully convolutional networks 
% and recurrent neural network by researchers at Notre Dame last year could apply 
% here \cite{fcnn_and_rnn_3d_bio_2016}. Some other interesting papers 
% \cite{CNN_Segmentation_Medical_Imaging_2017, Lai_2015}

\subsection{The Multiple Instance Learner/Classifier}

We will adapt MIL techniques developed for mammogram classification developed by
Lou et. al \red{add ref}. Their implementation feeds the 6 by 6 by 256 output of the last
convolutional layer of AlexNet as instances that are then consumed by three
different MIL methods/losses \red{which is it? a loss or a method? where is the backprop}.

We will use the same architecture and first use 2D instances to look for 
lung cancer. If time permits, we will adapt AlexNet to instead output 3D, voxel
based convolutions as the output instances for MIL methods. We particularly like
this AlexNet variant because, amongst the team, we own and have access to two
NVidia 970 GPUs.

\subsection{Data Augmentation}

One major challenge we will face will be the fact that we have a very small 
collection of data.
\temp{TODO:...}

\section{Datasets}
The number of lung CT scans available to us is very low. The total number of examples
available are many orders of magn{}itudes smaller than the size of datasets for modern,
state of the art classification challenges such as ImageNet or MSCOCO.

Listed below are the datsets we will leverage to train our classification/nodule
extraction model.

\begin{center}
\begin{tabular}{lll}
  \toprule
  Dataset & \# CT scans & Label Type\\
  \midrule
  Kaggle&1398&Cancer/No-cancer\\
  LIDC-IDRI&1018&Nodule annotation\\
  SPIE&80&Nodule annotation\\
  NLST&?&?\\
\end{tabular}
\end{center}

In the dataset above, each CT scan is a stack of 
approximately 200 2D slices.

The Kaggle dataset will be obtained from the Kaggle Data Science Bowl 2017
competition. The dataset consists of over 1000 CT scans in DICOM format labeled
by a cancer/no-cancer binary. We will use this data train our downstream
 MIL classifier.

The Lung Image Database Consortium (LIDC-IDRI) dataset is the dataset used in 
the 2016 LUNA challenge.The dataset consists of over 1000 CT scans where
each nodule location and radius is annotated for each 2D slice. The dataset was
annotated by 4 radiologists and nodules are annotated with varying levels of 
agreement. We will use this dataset to train our upstream region/nodule 
proposal network.

The SPIE dataset is a minuscule but have a few important examples where nodules
are both annotated and labeled benign or malignant. This dataset may help us 
validate and visualize the convolutions happening in both the upstream
region/nodule proposal and downstream MIL classifier.

We suspect that the National Lung Screening Trial (NLST) dataset will contain 
about another 1000 CT scans that will either be labeled with the 
cancer/no-cancer or nodule annotations. We have applied for and are
currently waiting for access to the NLST data.

\section{Evaluating our results}

\subsection{Our goals}
We have two primary goals in this investigation. First we want to know if region
proposal and implicit instance weighting aid multiple instance learning. Second,
we want to investigate the how leveraging depth information from our data helps
with inference with biomedical imaging.

\subsection{How will we know what is going on?}
\temp{
  - visualize 2D convolutions in both upstream and downstream nets \\
  - visualize 3D convs if we get to it in the downstream MIL net \\
  - Use SPIE data to see how malignant vs benign tumors are `seen' by our network\\
}


\subsection{Quantifying our results}
\temp{
  - COMPARE FASTER R-CNN with U-NET and other segmentation methods\\
  - Figure out how the upstream region output effects downstream classifier\\
  - Compare direct MIL classification with region enhanced MIL classification\\
}

\section{About us}
\temp{buncha kiddos who have no idea about anything}



\section{Questions and challenges}
\begin{enumerate}
	\item What will the output of node-extraction be like?
	\item What kind of novelty will we bring to the table?
	\item How will we deal with the small dataset(s)?
\end{enumerate}

\bibliographystyle{unsrt}%Used BibTeX style is unsrt
\bibliography{proposal}
\end{document}