\documentclass[twocolumn,10pt]{article}
\usepackage[T1]{fontenc}
\usepackage[utf8]{inputenc}
\usepackage{amsmath}
%\usepackage[english]{babel}
%\usepackage{biblatex}
\usepackage[margin=.75in]{geometry}
\setlength{\columnsep}{33pt}
\setcounter{secnumdepth}{2}
\usepackage{minipage-marginpar}
\newcommand\vfilbreak[1]{\vskip 0pt plus #1 \penalty-200 \vskip 0pt plus -#1}
\newenvironment{mpmp}[1]
               {\begin{minipagewithmarginpars}{#1}}
               {\end{minipagewithmarginpars}}
\usepackage{morefloats}
\usepackage{booktabs}
\usepackage{color}
\usepackage{graphicx}

\title{Lung Cancer Detection Milestone}
\author{Jay DeStories, Jason Fan, Alex Tong}
\date{March 2017}

\newcommand{\red}[1]{{\color{red}#1}}
\newcommand{\temp}[1]{{\red{#1}\\}}
\renewcommand{\b}{\boldsymbol}

\usepackage{enumitem}

\begin{document}

\maketitle
\section{Introduction}

In this document, we record the intermediary results for the Mult-Instance
Network Lung Cancer Detection project.

Please refer to project proposal for references to the proposed pipeline and
techniques.

\subsection{Things we didn't expect to be so difficult}

Our team found the following two probelems more difficult than expected:
\begin{enumerate}[noitemsep]
  \item Getting working implementation of AlexNet and learning TensorFlow
  \item Pre-processing Data and working with DICOM images
\end{enumerate}

\subsection{Problems with Pre-processing}

With the LUNA and Kaggle Dicom data amounts to almost a terabyte of storage, we
made some mistakes with data processing and had to process the Kaggle Data Twice.

We had trouble converting Hounsville Units (hu), the unit of measurement for DICOM
slice densities, to a sensible numeric value a traditional image classiication 
network would be able to consume. 

Hounsville Units measure density and map different parts of the human bodies to
a high range of values, we experimented with the DICOM slices to find appropriate
thresholded minimum and maximum values so that resulting images retained enough
contrast.

\temp{INSERT HISTOGRAM OF HU UNITS HERE}\\
\temp{INSERT DICOM LUNG without morph closing with low contrast...}\\
\temp{INSERT Pre-Processed image with segmented lung and threshholded values}\\

\subsection{From AlexNet in Keras to VGG in TensorFlow}

The Multi-Instance Network from the University of
California Irvine (UCI) researchers was implemented with a custom, unmaintained
version of Keras that we found, albeit too late, to be untenable. 

We decided to move to using TensorFlow for clearer documentation and more support
from the online community. At first, we found TensorFlow difficult to understand,
but we eventually found a simple enough implementation of VGG to work with.
(From: \texttt{github.com/machrisaa/tensorflow-vgg})

\section{Pipeline Progress}
We have implemented the following:
\begin{enumerate}[noitemsep]
  \item Memory concious RGB CNN to Grayscale CNN conversion
  \item Feature extraction with pre-trained VGG-16
  \item Feature extraction with pre-trained VGG-19
  \item Multi-Instance 2-class logistic regression
  \item Boosted decision tree
\end{enumerate}

\subsection{RGB to Grayscale conversion}
The Keras implementation of Multi-Instance Network feeds the first convolutional
an image of shape (224,224,3) in which the black and white dicom image is
copied once onto each Blue, Red and Green channel.

We attempted to mimick this technique with the stock VGG implementation to extract
slice-wise image features from the last fully convolutional layer and found that
we could not fit an entire lung volume into VRAM.

We noticed that for weight vector $\b w$ along the channels axis of a convolution, the
result of the convolution with pixel vector $\b x$ where each channel has value
$\hat x$, channelwise mean $\b \mu$ and bias
term $b$ was,

$$\b w \ast (\b x - \b \mu) + b = (\b w \ast \b x) + (b - \b w \ast \b \mu)
= \hat x \|\b w\|_1 + (b - \b w \ast \b \mu)$$

Which meant that we could eliminate the redundant 3-channel input dimension and
make any convolutional neural network effectively consume grayscale images by
 manipulating pretrained weights and biases for the first layer. We simply sum
 the convolution along the axis of the input channels dimension and subtract the
 term $(\b w \ast \b \mu)$ from the bias.

 We did this for both VGG-19 and VGG-19 and were able to forward pass entire
 lung volumes to extract slice-wise features.

\subsection{VGG feature extraction}
\temp{ADD VGG AS A CITATION}
We do the abovementioned RGB to Grayscale conversion for both VGG-19 and VGG-19 
and were able to reduce VRAM usage and forward pass entire lung volumes to 
extract slice-wise features. A feature tensor of shape (7, 7, 512) is extracted
for each slice, where each of the 49 receptive fields is represented by a feature
vector of length 512.

The resulting bag of features that is then consumed by the
Multi Instance 2-class Logistic Regression and Boosted Decision Tree Models,
is a tensor of shape (60, 7, 7, 512).

\subsection{Multi Instance 2-class logistic regression}
\temp{ADD MIL CITATION}
Let $F$ be a set of $N$ feature vectors that represent the instances of a given
lung slice. 
Then $\b r$, the vector of activations, is defined by elementwise by
$$r_i = \b w^T \b f_i + b$$.
Where $f_i$ and $r_i$ is the $i$-th elemebt of $\b r$ and $F$ respectively.

We then use $\b r $ to predict the probability of cancer $p(y = 1)$ where
$$t = p(y = 1) = \max_{r_i} \ \sigma(r_i)$$

The authors of the Multi-Instance proposed the following per example loss functions
to regularize hyperperameters at training time.

Normal softmax loss:
$$L_{max} = -\log(t^y + (1-t) ^{(1-y)})$$

Sparse softmax loss:
$$L_{max} = -\log(\lambda_t(t^y + (1-t) ^{(1-y)}) + \lambda_r\|\b r\|)$$

$$$$
\subsection {Boosted decision tree with XGBoost}

\section{Intermediary results}
\end{document}
